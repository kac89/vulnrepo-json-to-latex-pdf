
\documentclass[12pt]{article}
\author{&researcher;}
\title{&report_name;}
\date{\today}
\usepackage[spaces,hyphens]{url}
\usepackage{graphicx}
\usepackage[a4paper, total={17cm,26cm}]{geometry}
\begin{document}

\maketitle

&report_logo;
Report ID: &report_id;

Report Version: &report_version;
\newpage

\tableofcontents

\newpage

\section{Scope}

test

\section{Statistics and Risk}

\begin{center}
\begin{tabular}{||c c||}
 \hline
 Severity & Number \\ [0.5ex] 
 \hline\hline
 Critical & &critical_len; \\ 
 \hline
 High & &high_len; \\
 \hline
 Medium & &medium_len; \\
 \hline
 Low & &low_len; \\
 \hline
 Info & &info_len; \\
 \hline
 Total & &severity_total; \\ [1ex] 
 \hline
\end{tabular}
\end{center}

The risk of application security vulnerabilities discovered during an assessment will be rated according to a custom-tailored version the OWASP Risk Rating Methodology. Risk severity is determined based on the estimated technical and business impact of the vulnerability, and on the estimated likelihood of the vulnerability being exploited:\\
\begin{center}
\begin{tabular}{|c|c|c|c|c|}\hline\hline
   \multicolumn{5}{|c|}{Overall Risk Severity}\\\hline\hline
      Impact & HIGH & Medium & High & Critical \\\cline{1-5}
	       & MEDIUM & Low & Medium & High \\\cline{1-5}
	       & LOW & Info & Low & Medium \\\cline{1-5}
                &   & LOW & MEDIUM & HIGH \\\cline{1-5}
   \multicolumn{5}{|c|}{Likelihood}\\\hline\hline
\end{tabular}
\end{center}

Our Risk rating is based on this calculation: Risk = Likelihood * Impact.

\section{Issues (&severity_total;)}

&report_issues;

\section{Researcher}

&researcher;

\end{document}